%%%%%%%%%%%%%%%%%%%%%%%%%%%%%%%%%%%%%%%%%
% Awesome Resume/CV
% XeLaTeX Template
% Version 1.1 (9/1/2016)
%
% This template has been downloaded from:
% http://www.LaTeXTemplates.com
%
% Original author:
% Claud D. Park (posquit0.bj@gmail.com) with modifications by 
% Vel (vel@latextemplates.com)
%
% License:
% CC BY-NC-SA 3.0 (http://creativecommons.org/licenses/by-nc-sa/3.0/)
%
% Important note:
% This template must be compiled with XeLaTeX, the below lines will ensure this
%!TEX TS-program = xelatex
%!TEX encoding = UTF-8 Unicode
%
%%%%%%%%%%%%%%%%%%%%%%%%%%%%%%%%%%%%%%%%%

%----------------------------------------------------------------------------------------
%	PACKAGES AND OTHER DOCUMENT CONFIGURATIONS
%----------------------------------------------------------------------------------------

\documentclass[10pt, letterpaper]{awesome-cv} % A4 paper size by default, use 'letterpaper' for US letter

\geometry{left=1.5cm, top=1cm, right=1.5cm, bottom=1cm, footskip=.5cm} % Configure page margins with geometry

\fontdir[fonts/] % Specify the location of the included fonts

% Color for highlights
\colorlet{awesome}{awesome-red} % Default colors include: awesome-emerald, awesome-skyblue, awesome-red, awesome-pink, awesome-orange, awesome-nephritis, awesome-concrete, awesome-darknight
%\definecolor{awesome}{HTML}{CA63A8} % Uncomment if you would like to specify your own color

% Colors for text - uncomment and modify
%\definecolor{darktext}{HTML}{414141}
%\definecolor{text}{HTML}{414141}
%\definecolor{graytext}{HTML}{414141}
%\definecolor{lighttext}{HTML}{414141}

\renewcommand{\acvHeaderSocialSep}{\quad\textbar\quad} % If you would like to change the social information separator from a pipe (|) to something else

%----------------------------------------------------------------------------------------
%	PERSONAL INFORMATION
%	Comment any of the lines below if they are not required
%----------------------------------------------------------------------------------------

\name{Jaejun Brandon}{Lee}
%\address{Toronto, ON, Canada}
\mobile{(+1) 204-698-1771}
\email{ljj7975@gmail.com}
\linkedin{ljj7975}
\github{ljj7975}
\homepage{ljj7975.github.io}
%\position{PhD in computer science {\enskip\cdotp\enskip} Computational Biology {\enskip\cdotp\enskip} Cancer genomics {\enskip\cdotp\enskip} Machine learning} % Your expertise/fields
\position{Computer Vision {\enskip\cdotp\enskip} Natural Language Processing {\enskip\cdotp\enskip} Speech Recognition {\enskip\cdotp\enskip} Robotics}
%\quote{``Make the change that you want to see in the world."} % A quote or statement

\makecvfooter{\today}{Jaejun Brandon Lee}{\thepage} % Specify the letter footer with 3 arguments: (<left>, <center>, <right>), leave any of these blank if they are not needed

%----------------------------------------------------------------------------------------

\begin{document}

\makecvheader % Print the header

%----------------------------------------------------------------------------------------
%	CV/RESUME CONTENT
%	Each section is imported separately, open each file in turn to modify content
%----------------------------------------------------------------------------------------
\begin{cvletter}
\lettersection{Introduction}
\vspace*{-0.05cm}
Drawing from seven years of research experience across computer vision, speech recognition, natural language processing, and robotics, I believe that multimodal representation learning is not only effective in capturing knowledge about our world but also in controlling and interpreting a model’s internal representations. Therefore, \textbf{I am particularly interested in building a vision system that leverages language for effective communication with the user and efficient adaptation to dynamic environments}.
\end{cvletter}

%\href{https://github.com/ljj7975/ljj7975.github.io/tree/main/blog/research#what-is-my-interest}{\textbf{I am passionate about harnessing multimodality to establish an accurate representation of our world with greater interpretability, thereby fostering efficient domain adaptation.}. {\small \faLink}}

%----------------------------------------------------------------------------------------
%	SECTION TITLE
%----------------------------------------------------------------------------------------

\cvsection{Education}

%----------------------------------------------------------------------------------------
%	SECTION CONTENT
%----------------------------------------------------------------------------------------

\begin{cventries}

%------------------------------------------------

% \vspace*{-0.15cm}

\cventry
{Master of Mathematics in Computer Science, Advisor: Dr. Jimmy Lin} % Degree
{University of Waterloo} % Institution
{Waterloo, ON, Canada} % Location
{Jun. 2020} % Date(s)
{ % Description(s) bullet points
\begin{cvitems}
\item {Thesis: Honkling: In-Browser Personalization for Ubiquitous Keyword Spotting}
\item {Research area: Information retrieval and deep learning (natural language processing \& speech recognition)}
\item {Cumulative GPA : 88.75 \% (3.90 / 4.00)}
\end{cvitems}
}
\cventry
{Bachelor of Computer Science} % Degree
{University of Waterloo} % Institution
{Toronto, ON, Canada} % Location
{Oct. 2018} % Date(s)
{ % Description(s) bullet points
\begin{cvitems}
\item {Completed co-operative program and graduated with distinction}
\item {Recipient of President's Scholarship and Faculty of Mathematics Scholarship}
\item {Cumulative GPA : 81.18 \% (3.70 / 4.00)}
\end{cvitems}
}

\end{cventries}
\vspace*{-0.25cm}
%----------------------------------------------------------------------------------------
%	SECTION TITLE
%----------------------------------------------------------------------------------------

\cvsection{Scholarships}
\vspace*{-0.25cm}
%----------------------------------------------------------------------------------------
%	INTERNATIONAL SUBSECTION
%----------------------------------------------------------------------------------------

\iffalse


\cvsubsection{International}

%------------------------------------------------

\begin{cvhonors}
\cvhonor
{Award} % Award
{American Society of Hematology, Abstract Achievement Award} % Event
{value of USD 500} % Location
{Dec, 2022} % Date(s)

\end{cvhonors}

%----------------------------------------------------------------------------------------
%	DOMESTIC SUBSECTION
%----------------------------------------------------------------------------------------

\cvsubsection{Domestic}

%------------------------------------------------

\begin{cvhonors}

\cvhonor
{Award} % Award
{Society of Chinese Bioscientists in America, Toronto Chapter, Best Paper award (honourable mention)} % Event
{value of CAD 100} % Location
{Feb, 2019} % Date(s)


\end{cvhonors}
\fi


\begin{cvhonors}

\cvhonor
{Scholarship} % Award
{Faculty of Mathematics Scholarship, University of Waterloo} % Event
{value of CAD 25,000} % Location
{2013-2018} % Date(s)

\cvhonor
{Scholarship} % Award
{President’s Scholarship, University of Waterloo} % Event
{value of CAD 2,000} % Location
{2013} % Date(s)

\cvhonor
{Scholarship} % Award
{Winnipeg North Rotary Club Scholarship} % Event
{value of CAD 5,000} % Location
{2013} % Date(s)

\cvhonor
{Scholarship} % Award
{Loblaw Scholarship} % Event
{value of CAD 1,500} % Location
{2013} % Date(s)

\end{cvhonors}

%----------------------------------------------------------------------------------------
%	SECTION TITLE
%----------------------------------------------------------------------------------------

\cvsection{Experience}

%----------------------------------------------------------------------------------------
%	SECTION CONTENT
%----------------------------------------------------------------------------------------
\vspace*{0.05cm}
\begin{cventries}

%------------------------------------------------    
\cventry
{Epson Canada - Computer Vision \& Robotics Lab} % Organization
{Research Scientist} % Job title
{Markham, ON, Canada} % Location
{Mar. 2023 - Present} % Date(s)
{ % Description(s) of tasks/responsibilities\\
\begin{cvitems}
\item {Developing a customizable and interpretable 2D detection system for non-expert users by leveraging a large language model.}
\item {Utilizing a novel open-set object segmentation method, developed a prompt-based keypoint detection pipeline that remains robust \\
in multi-instance scenarios (one first-author paper under review for ECCV 2024).}
\end{cvitems}
}
\cventry
{Roboeye.ai} % Organization
{Lead Research Scientist} % Job title
{Toronto, ON, Canada} % Location
{Jul. 2021 - Mar. 2023} % Date(s)
{ % Description(s) of tasks/responsibilities\\
\begin{cvitems}
\item {Developed a real-time (<1 sec) 6D pose estimation pipeline integrating latest computer vision techniques.}
\begin{itemize}[label=$\cdot$,leftmargin=0.7em]
\item{Point cloud reconstruction + instance segmentation (Mask R-CNN \& DetectoRS) + pose estimation (FCGF-based RANSAC \& PVN3D) + \\
pose refinement (ICP) + detection filtering (3D NMS).}
\end{itemize}
\item {Led a team of 20+ engineers in deploying and maintaining 50+ bin-picking systems for continuous operation without failure.}
\end{cvitems}
}
\cventry
{Roboeye.ai} % Organization
{Research Scientist} % Job title
{Toronto, ON, Canada} % Location
{Mar. 2020 - Jul. 2021} % Date(s)
{ % Description(s) of tasks/responsibilities\\
\begin{cvitems}
\item {Developed an intuitive vision-driven bin-picking solution by leveraging a 6D pose estimation pipeline.}
\begin{itemize}[label=$\cdot$,leftmargin=0.7em]
\item {Fully automated online model training system using PyTorch, NVIDIA Isaac Sim, OpenCV, and AWS.}
\item {C++ application designed for bin-picking tasks using ROS, Qt5, Protobuf, OpenCV, and PCL.}
\item {Online object detection performance tracking system using AWS, Docker, W\&B, Django.}
\end{itemize}
\end{cvitems}
}
\cventry
{Mozilla - Emerging Technologies Team} % Organization
{Research Scientist} % Job title
{Remote} % Location
{Mar. 2020 - Oct. 2020} % Date(s)
{ % Description(s) of tasks/responsibilities\\
\begin{cvitems}
\item {Developed a wake-word detection system for Firefox, \href{https://github.com/castorini/howl}{Howl {\small \faGithub}}, publishing a first-author paper at an EMNLP workshop [\hyperlink{howl:EMNLP}{3}].}
\item {Integrated Howl with \href{https://github.com/mozilla-extensions/firefox-voice}{Firefox Voice {\small \faGithub}} to provide a completely hands-free experience to over 8,000 users.}
\end{cvitems}
}
\cventry
{Samsung Research America - Visual Display Intelligence Lab} % Organization
{Research Scientist} % Job title
{Mountain View, CA, USA} % Location
{Apr. 2019 - Mar. 2020} % Date(s)
{ % Description(s) of tasks/responsibilities\\
\begin{cvitems}
\item {Developed a novel co-clustering algorithm leveraging GANs, resulting in a first-author paper at ICME 2021 [\hyperlink{CI-GAN:ICME}{2}] and \\the filing of two related patents [\hyperlink{CI-GAN:International}{9}, \hyperlink{CI-GAN:US}{10}].}
\begin{itemize}[label=$\cdot$,leftmargin=0.7em]
\item {Jointly learns disentangled representations of dual data dimensions and their underlying interrelation in the correlation space.}
\end{itemize}
\item {Implemented user-centric TV program recommendation by analyzing watch history.}
\end{cvitems}
}
\cventry
{University of Waterloo - Data Systems Group} % Organization
{Graduate Student Researcher} % Job title
{Waterloo, ON, Canada} % Location
{Sep. 2018 - Dec. 2019} % Date(s)
{ % Description(s) of tasks/responsibilities\\
\begin{cvitems}
\item {\href{https://github.com/castorini/honkling}{Personalized Keyword Spotting System {\small \faGithub}} -- two first-author papers at EMNLP 2019 [\hyperlink{honkling:EMNLP}{6}] and IUI 2019 [\hyperlink{honkling:IUI}{8}].}
\begin{itemize}[label=$\cdot$,leftmargin=0.7em]
\item {Implemented keyword spotting with convolutional neural networks in pure JavaScript that runs in any standards-compliant browser.}
\item {Applied fine-tuning based accent adaptation and studied its efficiency in the browser.}
\end{itemize}
\item {PEfficient Parameter Fine-Tuning of Large Language Models -- two papers at ACL 2020 [\hyperlink{show:ACL}{4}, \hyperlink{DeeBERT:ACL}{5}].}
\begin{itemize}[label=$\cdot$,leftmargin=0.7em]
\item {Developed memory/latency reduction techniques and investigated the effects of freezing various layers for language models (BERT).}
\end{itemize}
\end{cvitems}
}
\cventry
{University of Waterloo} % Organization
{Undergraduate Research Assistant} % Job title
{Waterloo, ON, Canada} % Location
{May. 2018 - Aug. 2018} % Date(s)
{ % Description(s) of tasks/responsibilities\\
\begin{cvitems}
\item {Studied the suitability of JavaScript as an environment for deep learning execution.}
\end{cvitems}
}
\cventry
{Meta (Facebook) - Dynamic Ads Infrastructure} % Organization
{Software Engineer Intern} % Job title
{Menlo Park, CA, USA} % Location
{Jan. 2018 - Apr. 2018} % Date(s)
{ % Description(s) of tasks/responsibilities\\
\begin{cvitems}
\item {Applied KNN algorithms on product-level and user-level embeddings to enhance the quality of personalized advertisements.}
\item {Redesigned the advertisements selection pipeline to retrieve user embeddings at an earlier stage, reducing loading time by 7\%.}
\end{cvitems}
}
\cventry
{University of Waterloo} % Organization
{Undergraduate Research Assistant} % Job title
{Waterloo, ON, Canada} % Location
{Sep. 2017 - Dec. 2017} % Date(s)
{ % Description(s) of tasks/responsibilities\\
\begin{cvitems}
\item {Implemented \href{https://github.com/ljj7975/CachedRDDReportGenerator}{an RDD usage report generator for Spark {\small \faGithub}} and analyzed the impact of caching replacement policies on performance.}
\end{cvitems}
}
\cventry
{University of Waterloo} % Organization
{Undergraduate Research Assistant} % Job title
{Waterloo, ON, Canada} % Location
{Sep. 2017 - Dec. 2017} % Date(s)
{ % Description(s) of tasks/responsibilities\\
\begin{cvitems}
\item {Analyzed latency and throughputs of Apache Storm and Spark Streaming; benchmarked against TPCx-IoT specifications.}
\end{cvitems}
}
\cventry
{Uber - Complex Data Processing / Spark Team} % Organization
{Software Engineer Intern} % Job title
{Palo Alto, CA, USA} % Location
{May. 2017 - Aug. 2017} % Date(s)
{ % Description(s) of tasks/responsibilities\\
\begin{cvitems}
\item {Integrated TensorFlowOnSpark on Uber infrastructure and evaluated its stability.}
\item {Transformed MLlib pipeline into a Spark job with TensorFlow; reduced training time from 33 to 3 hours.}
\end{cvitems}
}
\cventry
{Zynga Inc - Central Technology Organization} % Organization
{Software Engineer Intern} % Job title
{Toronto, ON, Canada} % Location
{Aug. 2016 - Dec. 2016} % Date(s)
{ % Description(s) of tasks/responsibilities\\
\begin{cvitems}
\item {Developed a new architecture for the internal search system.}
\begin{itemize}[label=$\cdot$,leftmargin=0.7em]
\item {Improved data integrity led to 30\% increase in search usage (Amazon Elasticsearch, Amazon Kinesis Streams and Amazon SQS).}
\end{itemize}
\end{cvitems}
}
\cventry
{SAP - Emerging Technologies Team} % Organization
{Software Engineer Intern} % Job title
{Waterloo, ON, Canada} % Location
{Jan. 2016 - Apr. 2016} % Date(s)
{ % Description(s) of tasks/responsibilities\\
\begin{cvitems}
\item {Designed and developed a distributed SQLA back-end system with support for the OData protocol.}
\item {Integrated Robot framework, an automated testing tool, to reduce QA cycle from 3 days to 4 hours.}
\end{cvitems}
}
\cventry
{Mozzaz Corporation} % Organization
{Software Engineer Intern} % Job title
{Toronto, ON, Canada} % Location
{May. 2017 - Aug. 2017} % Date(s)
{ % Description(s) of tasks/responsibilities\\
\begin{cvitems}
\item {Developed a cross-platform web application using Cordova and Angular.js; performed back-end development with C\#.}
\end{cvitems}
}

\end{cventries}

%----------------------------------------------------------------------------------------
%	SECTION TITLE
%----------------------------------------------------------------------------------------

\cvsection{Publications and Patents}\vspace*{-0.40cm}
\hfill \descriptionstyle{* equal contribution}\vspace*{-0.30cm}
%----------------------------------------------------------------------------------------
%	SECTION CONTENT
%----------------------------------------------------------------------------------------

\cvsubsection{Publications}\vspace*{-0.15cm}
\\
%\begin{cvitems}\vspace*{-0.15cm}
\begin{cvenumerate}[1]
\item {Hwayeon Danielle Shin*, \textcolor{gray!99!black}{\textbf{Jaejun Lee*}}, Imam Kassam, Federica Guccini. Mining Public Voices: Analyzing Suicide Related Thoughts and Behaviors in YouTube Videos and Comments Using Topic Modeling. \textbf{\textit{FHLIP}}, 2025}
\item {Tomasz Palczewski*, \textcolor{gray!99!black}{\textbf{Jaejun Lee*}}, Lenin Mookiah*. Production-Ready Applied Deep Learning. \textbf{\textit{Packt Publishing}}, ISBN: 9781803238050, 1803238054, 2022}
\item \hypertarget{CI-GAN:ICME}{\textcolor{gray!99!black}{\textbf{Jaejun Lee}}, Hyun Chul Lee, Tomasz Palczewski. CI-GAN: Co-Clustering By Information Maximizing Generative Adversarial Networks. \textbf{\textit{ICME}}, 2021}
\item \hypertarget{howl:EMNLP}{Raphael Tang*, \textcolor{gray!99!black}{\textbf{Jaejun Lee*}}, Afsaneh Razi, Julia Cambre, Ian Bicking, Jofish Kaye, Jimmy Lin. Howl: A Deployed, Open-Source Wake Word Detection System. \textbf{\textit{EMNLP-NLPOSS}}, 2020}
\item \hypertarget{show:ACL}{Raphael Tang, \textcolor{gray!99!black}{\textbf{Jaejun Lee}}, Ji Xin, Xinyu Liu, Yaoliang Yu, Jimmy Lin. Showing Your Work Doesn't Always Work. \textbf{\textit{ACL}}, 2020}
\item \hypertarget{DeeBERT:ACL}{Ji Xin, Raphael Tang, \textcolor{gray!99!black}{\textbf{Jaejun Lee}}, Yaoliang Yu, Jimmy Lin. DeeBERT: Dynamic Early Exiting for Accelerating BERT Inference. \textbf{\textit{ACL}}, 2020}
\newpage
\item \hypertarget{honkling:EMNLP}{\textcolor{gray!99!black}{\textbf{Jaejun Lee}}, Raphael Tang, Jimmy Lin. Honkling: In-Browser Personalization for Ubiquitous Keyword Spotting. \textbf{\textit{EMNLP-IJCNLP}}, 2019}
\item {Ryan Clancy, \textcolor{gray!99!black}{\textbf{Jaejun Lee}}, Zeynep Akkalyoncu Yilmaz, Jimmy Lin. Information Retrieval Meets Scalable Text Analytics: Solr Integration with Spark. \textbf{\textit{SIGIR}}, 2019}
\item \hypertarget{honkling:IUI}{\textcolor{gray!99!black}{\textbf{Jaejun Lee}}, Raphael Tang, Jimmy Lin. Universal Voice-Enabled User Interfaces using JavaScript. \textbf{\textit{IUI}}, 2019}
\end{cvenumerate}
\hfill

\cvsubsection{Patents}
\\
\begin{cvenumerate}[10]\vspace*{-0.15cm}
\item \hypertarget{CI-GAN:International}{\textcolor{gray!99!black}{\textbf{Jaejun Lee}}, Hyun Chul Lee, Tomasz Palczewski. Co-Informatic Generative Adversarial Networks for Efficient Data \\Co-Clustering. \textbf{\textit{International Patent}}, Pub. WO/2021/066530, 2021}
\item \hypertarget{CI-GAN:US}{\textcolor{gray!99!black}{\textbf{Jaejun Lee}}, Hyun Chul Lee, Tomasz Palczewski. Co-Informatic Generative Adversarial Networks for Efficient Data \\Co-Clustering. \textbf{\textit{US Patent}}, Pub. 20210097372, 2021}
\end{cvenumerate}
\hfill

\cvsubsection{Manuscripts}
\\
\begin{cvenumerate}[12]\vspace*{-0.15cm}
\item \hypertarget{Elsa:arXiv}{\textcolor{gray!99!black}{\textbf{Jaejun Lee}}, Raphael Tang, Jimmy Lin. What Would Elsa Do? Freezing Layers During Transformer Fine-Tuning. \textit{arXiv}: 1911.03090, 2019}
\item \hypertarget{JavaScript:arXiv}{\textcolor{gray!99!black}{\textbf{Jaejun Lee}}, Raphael Tang, Jimmy Lin. JavaScript Convolutional Neural Networks for Keyword Spotting in the Browser: An Experimental Analysis. \textit{arXiv}: 1810.12859, 2018}
\end{cvenumerate}
\hfill

\cvsubsection{Under Review}
\\
\begin{cvenumerate}[14]\vspace*{-0.15cm}
\item \hypertarget{LAM:TAI}{\textcolor{gray!99!black}{\textbf{Jaejun Lee}}, Aristide Tossou, Sandra Wang, Dibyendu Mukherjee. LAM: Few-Shot Random Keypoint Detection in Any Scene through Open-set
Object Segmentation. \textbf{\textit{TAI}}}
\item \hypertarget{TIDES:NeurIPS}{\textcolor{gray!99!black}{\textbf{Jaejun Lee*}}, Shin Nishimura, Sahar Ghavidel, Allen Tao, Dibyendu Mukherjee. TIDES: Training-free Instance Detection from Semantics. \textbf{\textit{NeurIPS}}, 2025}
\end{cvenumerate}
%----------------------------------------------------------------------------------------
%	SECTION TITLE
%----------------------------------------------------------------------------------------

\cvsection{Presentation}
%----------------------------------------------------------------------------------------
%	SECTION CONTENT
%----------------------------------------------------------------------------------------
\vspace*{0.05cm}
\begin{cventries}
\cventry
{Poster Presentation} % Role
{2024 American Medical Informatics Association (AMIA) Annual Symposium} % Event
{San Francisco, CA, USA} % Location
{Nov., 2024} % Date(s)
{ % Description(s)
\begin{cvitems}
\item {Analyzing YouTube Videos on Suicide-Related Thoughts and Behaviours: A Study Using Topic Modeling and Discourse Analysis.}
\end{cvitems}
}
\cventry
{Oral Presentation} % Role
{2024 IHPME Research and Impact Day} % Event
{Toronto, ON, Canada} % Location
{Apr., 2024} % Date(s)
{ % Description(s)
\begin{cvitems}
\item {Analyzing YouTube Videos on Suicide-Related Thoughts and Behaviours: A Study Using Topic Modeling and Discourse Analysis.}
\end{cvitems}
}
\cventry
{Oral Presentation} % Role
{Epson's Global Information Sharing Meeting} % Event
{Virtual} % Location
{Apr., 2024} % Date(s)
{ % Description(s)
\begin{cvitems}
\item {Advances in Large Vision and Language Models Driven by Prompt Engineering for Efficient Domain Adaptation.}
\end{cvitems}
}
\cventry
{Poster Presentation} % Role
{2024 Annual Meeting of the Society for Digital Mental Health} % Event
{Virtual} % Location
{Apr., 2024} % Date(s)
{ % Description(s)
\begin{cvitems}
\item {Analyzing YouTube Videos on Suicide-Related Thoughts and Behaviours: A Study Using Topic Modeling and Discourse Analysis.}
\end{cvitems}
}
\cventry
{Oral Presentation} % Role
{Epson's Canadian Information Sharing Meeting} % Event
{Markham, ON, Canada} % Location
{Nov., 2023} % Date(s)
{ % Description(s)
\begin{cvitems}
\item {Enhancing 2D Object Detection Efficiency through Prompt Engineering.}
\end{cvitems}
}
\cventry
{Poster Presentation} % Role
{2nd Workshop for Natural Language Processing Open Source Software (NLP-OSS)} % Event
{Virtual} % Location
{Nov., 2020} % Date(s)
{ % Description(s)
\begin{cvitems}
\item {Howl: A Deployed, Open-Source Wake Word Detection System.}
\end{cvitems}
}
\cventry
{Poster Presentation} % Role
{2019 Conference on Empirical Methods in Natural Language Processing (EMNLP) and\newline 9th International Joint Conference on Natural Language Processing (IJCNLP)} % Event
{Hong Kong, China} % Location
{Nov., 2019} % Date(s)
{ % Description(s)
\begin{cvitems}
\item {Honkling: In-Browser Personalization for Ubiquitous Keyword Spotting.}
\end{cvitems}
}
\cventry
{Poster Presentation} % Role
{24th International Conference on Intelligent User Interfaces (IUI)} % Event
{Los Angeles, CA, USA} % Location
{Mar., 2019} % Date(s)
{ % Description(s)
\begin{cvitems}
\item {Universal Voice-Enabled User Interfaces using JavaScript.}
\end{cvitems}
}
\end{cventries}


%----------------------------------------------------------------------------------------
%	SECTION TITLE
%----------------------------------------------------------------------------------------

\cvsection{Professional Development}

%----------------------------------------------------------------------------------------
%	SECTION CONTENT
%----------------------------------------------------------------------------------------

\cvsubsection{Teaching Assistantship}
\begin{cventries}
\cventrynodesc
{University of Waterloo, Instructed by Prof. Bill Cowan} % Organization
{CS 452/652 -- Real-time Programming} % Job title
{Waterloo, ON, Canada} % Location
{Fall 2019} % Date(s)
\cventrynodesc
{University of Waterloo, Instructed by Prof. Edith Law} % Organization
{CS 480/680 -- Introduction to Machine Learning} % Job title
{Waterloo, ON, Canada} % Location
{Winter 2019} % Date(s)
\cventry
{University of Waterloo, Instructed by Prof. Jimmy Lin} % Organization
{CS 451/651 -- Data Intensive Distributed Computing} % Job title
{Waterloo, ON, Canada} % Location
{Fall 2018} % Date(s)
{ % Description(s) of tasks/responsibilities\\
\begin{cvitems}
\item {Led weekly discussion sections consisting of 10$\sim$20 students and held weekly office hours and help sessions.}
\item {Facilitated exams, graded problem sets and exams, and held exam preparation sessions.}
\end{cvitems}
}
\end{cventries}
\newpage
\cvsubsection{Academic Mentorship}
\begin{cventries}
\cventry
{Research Intern} % Organization
{Allen Tao} % Job title
{Markham, ON, Canada} % Location
{May. 2024 - Present} % Date(s)
{ % Description(s) of tasks/responsibilities\\
\begin{cvitems}
\item {Mentoring a research project on few-shot object detection.}
\end{cvitems}
}
\cventry
{Research Intern} % Organization
{Sandra Wang} % Job title
{Markham, ON, Canada} % Location
{Mar. 2023 - May. 2024} % Date(s)
{ % Description(s) of tasks/responsibilities\\
\begin{cvitems}
\item {Mentored research projects on 3D scene understanding and few-shot keypoint detection (one paper under review for AAAI 2025).}
\end{cvitems}
}
\cventry
{Undergraduate Research Assistant} % Organization
{Xinyu (Mavis) Liu} % Job title
{Waterloo, ON, Canada} % Location
{Jan. 2018 - Aug. 2018} % Date(s)
{ % Description(s) of tasks/responsibilities\\
\begin{cvitems}
\item {Mentored an undergraduate research project on keyword spotting.}
\end{cvitems}
}
\end{cventries}

%\newpage
%%----------------------------------------------------------------------------------------
%	SECTION TITLE
%----------------------------------------------------------------------------------------

\cvsection{Skills}

%----------------------------------------------------------------------------------------
%	SECTION CONTENT
%----------------------------------------------------------------------------------------

\begin{cvskills}

%------------------------------------------------

\cvskill
{Technical skills} % Category
{R, Perl, Python, Matlab, C++, Java, git, AWS (S3 and EC2), Unix/Linux, High-performance computing} % Skills
\cvskill
{Next-generation sequencing} % Category
{DNA (WGS, WXS, and Targeted)/RNA seq (WTS and Targeted), IP-based sequencing, single-cell seq (10x, Tapestri)} % Skills
\cvskill
{Other high-throughput platforms} % Category
{Various array platforms, Nanostring, and Optical Genome Mapping} % Skills
\end{cvskills}

%%----------------------------------------------------------------------------------------
%	SECTION TITLE
%----------------------------------------------------------------------------------------

\cvsection{Memberships}

%----------------------------------------------------------------------------------------
%	SECTION CONTENT
%----------------------------------------------------------------------------------------

\begin{cvhonors}

%------------------------------------------------

\cvhonor
{Member} % Position
{American Society of Hematology} % Committee
{} % Location
{2013-2019} % Date(s)
    
%------------------------------------------------

\cvhonor
{Member} % Position
{European Hematology Association} % Committee
{} % Location
{2016-Present} % Date(s)
%------------------------------------------------

\end{cvhonors}
%%----------------------------------------------------------------------------------------
%	SECTION TITLE
%----------------------------------------------------------------------------------------

\cvsection{References}

%----------------------------------------------------------------------------------------
%	SECTION CONTENT
%----------------------------------------------------------------------------------------

\begin{cventries}

\cventry
{} % Job title
{Professor. Zhaolei Zhang, PhD} % Organization
{} % Location
{} % Date(s)
{\begin{cvitems}\vspace*{-0.35cm}
\item {Department of Molecular Genetics, Department of Computer Science, Donnelly Centre for Cellular and Biomolecular Research, University of Toronto, Toronto, ON, Canada}
\item{Email: zhaolei.zhang@utoronto.ca}
\end{cvitems}}

\cventry
{} % Job title
{Professor. Dennis Dong-Hwan Kim, MD, PhD} % Organization
{} % Location
{} % Date(s)
{\begin{cvitems}\vspace*{-0.35cm}
\item {Division of Medical Oncology and Hematology, Allogeneic Blood and Marrow Transplant and Leukemia Program, Princess Margaret Cancer Centre, Toronto, ON, Canada}
\item{Email: dr.dennis.kim@uhn.ca}
\end{cvitems}}

%\cventry
%{Professor} % Job title
%{Dr. Seishi Ogawa, MD, PhD} % Organization
%{} % Location
%{} % Date(s)
%{\begin{cvitems}
%\item {Department of Pathology and Tumor biology, Kyoto University, Kyoto, Japan}
%\end{cvitems}}

\end{cventries}

%----------------------------------------------------------------------------------------

\end{document}